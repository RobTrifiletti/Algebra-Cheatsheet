\section{Gröbner baser (5)}
\subsection{Polynomier i flere variable (5.1)}
\label{Polynomier i flere variable (5.1)}
Vi husker hvordan vi definerede en \nameref{Polynomiumsring} i en variabel. Vi
udvider nu dette til en polynomiumsring i flere variable $R[\myto{X_1}{X_n}]$
som
\begin{equation*}
  R[\myto{X_1}{X_n}] = R[\N^n] = 
  \myset{f : \N^n \to R \mid f(v) = 0, \myabs{v} \gg 0}
\end{equation*}
hvor $v = (\myto{v_1}{v_n}) \in \N^n$ og $\myabs{v} = v_1 + \cdots + v_n$.

Et polynomium $f \in R[\myto{X_1}{X_n}]$ er altså det samme som en afbildning
\begin{equation*}
  f: \N^n \to R
\end{equation*}
der er ikke-nul for kun endeligt mange $v$.

Vi lader $X^v \in R[\N^n]$ være afbildningen
\begin{equation*}
  X^v(w) =  
  \left\{ 
  \begin{array}{l l}
    1 & \quad \text{if $v = w$}\\
    0 & \quad \text{if $v \neq w$}\\
  \end{array} \right.
\end{equation*}

Med denne notation kan vi nu skrive polynomier i flere variable, $f \in
R[\N^n]$, som en endelig sum
\begin{equation*}
  f = \sum_{v \in N^n} a_v X^v
\end{equation*}
hvor $a_v \in R$.

Vi definerer addition af to polynomier $f, g \in R[\N^n]$ på følgende vis
\begin{equation*}
  f + g = (f + g)(v) = f(v) + g(v)
\end{equation*}

og multiplikation $fg$ som den endelige sum
\begin{equation*}
  (fg)(v) = \sum_{v_1 + v_2 = v} f(v_1) g(v_2)
\end{equation*}
hvor $v_1, v_2 \in \N^n$.

Bemærk at
\begin{enumerate}[(i)]
  \item $0 \in R$ er det neutrale element for $+$.
  \item Afbildningen $X^{(0,0,\ldots,0)}$ der afbilder 0-vektoren i $\N^n$ til
  $1 \in R$ og alt andet til $0 \in R$ til at være det neutrale element for
  $\cdot$.
\end{enumerate}

Ved notationen $R[\myto{X_1}{X_n}]$ for $R[\N^n]$ menes
\begin{enumerate}[(i)]
  \item $X_1 = X^{(1,0,\ldots,0)}$
  \item $X_2 = X^{(0,1,\ldots,0)}$
  \item $X_n = X^{(0,0,\ldots,1)}$
\end{enumerate}

Et \textit{led} er et polynomium $rX^v \in R[\N^n]$, hvor $r \in
R\mysetminus{0}$ kaldes \textit{koefficienten}.

Som eksempel kan vi skrive polynomiet $f$ op på to ens måder, en formel og en
konkret. De er ækvivalente
\begin{align*}
  f &= 2X^{(0,0,0)} + 2X^{(1,0,3)} + X^{(2,1,0)} - X^{(0,1,1)} + 3X^{(1,1,1)}
  \in \Z[\N^3]\\
    &= 2 + 2TZ^3 + T^2 Y - YZ + 3TYZ \in \Z[T,Y,Z] 
\end{align*}
hvor $T = X^{(1,0,0)}, Y = X^{(0,1,0)}$ og $Z = X^{(0,0,1)}$. Det formelle
polynomium er nu udtrykt ved de 3 variable $T,Y,Z$.



\subsubsection{Termordning}
\label{Termordning}
Def. 5.1.2: Mængden $\N^n$, bestående af $n$-tupler af naturlige tal, har en
komponentvis addition $+$ med nul-vektoren $0 = (0,\ldots,0)$. En
\nameref{Partiel ordning} $\leq$ på $\N^n$ kaldes en termordning hvis
\begin{enumerate}[(i)]
  \item $\leq$ er en \nameref{Total ordning}.
  \item $0 \leq v$
  \item $v_1 \leq v_2 \Rightarrow v_1 + v \leq v_2 + v$
\end{enumerate}
$\forall v, v_1, v_2 \in N^n$.

\subsubsection{Leksikografisk ordning}
\label{Leksikografisk ordning}
Vi definerer den leksikografiske ordning $\leq_{lex}$ på $\N^n$ ved
\begin{equation*}
  (v_1,\ldots,v_n) \leq_{lex} (w_1,\ldots,w_n)
\end{equation*}
hvis en af følgende gælder
\begin{align*}
  &(v_1 < w_1) \quad \text{eller}\\
  &(v_1 = w_1) \text{ og } (v_2 < w_2) \quad \text{eller}\\
  &(v_1 = w_1) \text{ og } (v_2 = w_2) \text{ og } (v_3 < w_3) \quad \text{etc}
\end{align*}
Dette vil sige normal alfabetisk ordning. F.eks. $(1,1,3) \leq_{lex} (1,2,3)$
da $1 < 2$ og $(4,5,1) \leq_{lex} (4,5,3)$ da $1 < 3$. Den leksikografiske
ordning er en \nameref{Termordning}.

\subsubsection{Graderet leksikografisk ordning}
\label{Graderet leksikografisk ordning}
Lad $\myabs{v} = v_1 + \cdots + v_n$, hvor $v = (v_1,\ldots,v_n) \in \N^n$. Vi
definerer nu den graderede leksikografiske ordning ved
\begin{equation*}
  (v_1,\ldots,v_n) \leq_{grlex} (w_1,\ldots,w_n)
\end{equation*}
hvis
\begin{align*}
  &\myabs{v} < \myabs{w} \quad \text{eller}\\
  &\myabs{v} = \myabs{w} \text{ og } v \leq_{lex} w
\end{align*}
F.eks. $(2,1,1) \leq_{grlex} (1,2,3)$, da $(2 + 1 + 1 < 1 + 2 + 3)$, selvom\\
$(2,1,1) \geq_{lex} (1,2,3)$. Den graderede leksikografiske ordning er en
\nameref{Termordning}.

\subsubsection{Lemma 5.1.5 (Dickson)}
\label{Lemma 5.1.5 (Dickson)}
Lad $S$ være en delmængde af $\N^n$. Så er der en endelig mængde af vektorer
$v_1,\ldots,v_r \in S$ sådan at
\begin{equation*}
  S \subseteq (v_1 + \N^n) \cup \cdots \cup (v_r + \N^n)
\end{equation*}
hvor $v_i + \N^n = \myset{v_i + w \mid w \in \N^n}$ for en vektor $v_i \in
\N^n$.

\subsubsection{Korollar 5.1.7}
\label{Korollar 5.1.7}
En \nameref{Termordning} $\leq$ på $\N^n$ er en \nameref{Velordning}.

\subsection{Initialterm (5.2)}
\label{Initialterm (5.2)}
\subsubsection{Initialterm}
\label{Initialterm}
Def. 5.2.1: Lad
\begin{equation*}
  f = \sum_{v \in \N^n} a_v X^v
\end{equation*}
være et ikke-nul polynomium i $R[\N^n]$ og $\leq$ er \nameref{Termordning} på
$\N^n$.

Initialtermet af $f$ mht. $\leq$ er defineret som
\begin{equation*}
  in_{\leq}(f) = a_w X^w
\end{equation*}
hvor $w = max_{\leq}\myset{v \in \N^n \mid a_v \neq 0}$.

F.eks. Lad
\begin{equation*}
  f = X^2 + XY + Y + Y^3 + X^5 \in \Z[X,Y]
\end{equation*}
hvor $X = X^{(1,0)}$ og $Y = X^{(0,1)}$ i $\Z[\N^n]$. Derved er
\begin{equation*}
  f = X^{(2,0)} + X^{(1,1)} + X^{(0,1)} + X^{(0,3)} + X^{(5,0)} \in \Z[\N^2]
\end{equation*}
Lader vi nu $\leq = \leq_{lex}$ får vi
\begin{equation*}
  (5,0) \geq (2,0) \geq (1,1) \geq (0,3) \geq (0,1)
\end{equation*}
Efter denne orden skulle man skrive $f = X^5 + X^2 + XY + Y^3 + Y$. Derved er
\begin{equation*}
  in_{\leq}(f) = X^5
\end{equation*}

\subsubsection{Bemærkning 5.2.3}
\label{Bemaerkning 5.2.3}
Lad $R$ være et \nameref{Integritetsomraade} og $f,g \in R[\myto{X_1}{X_n}]$.
Så gælder
\begin{equation*}
  in_{\leq}(fg) = in_{\leq}(f)in_{\leq}(g)
\end{equation*}
Dette er en analogt med \nameref{Prop. 4.2.2}, for polynomier i en variabel.

\subsection{Divisionsalgoritmen (5.3)}
\label{Divisionsalgoritmen (5.3)}
\subsubsection{Prop. 5.3.1}
\label{Prop. 5.3.1}
Prop. 5.3.1: Hold en \nameref{Termordning} $\leq$ på $\N^n$ fast. Lad $f \in
R[\myto{X_1}{X_n}] \mysetminus{0}$ og antag at $\myto{f_1}{f_m} \in
R[\myto{X_1}{X_n}]$ er en sekvens af ikke-nul polynomier. Så eksisterer der
$a_1,\ldots,a_m, r \in R[\myto{X_1}{X_n}]$ sådan at
\begin{equation*}
  f = a_1 f_1 + \cdots + a_m f_m + r
\end{equation*}
hvor $r = 0$ eller ingen af leddene i $r$ er delelig med
$in_{\leq}(f_1),\ldots,in_{\leq}(f_n)$. Desuden er $in_{\leq}(a_i f_i) \leq
in_{\leq}(f)$ hvis $f_i \neq 0$. Se \nameref{Prop. 4.2.4} for algoritmen i en
variabel.

\subsubsection{Def. 5.3.2}
\label{Def. 5.3.2}
Antag $f \in R[\myto{X_1}{X_n}]$ og lad $F = (f_1,\ldots,f_m)$ være en sekvens
af ikke-nul polynomier i $R[\myto{X_1}{X_n}]$. Vi lader $f^F$ beskrive resten
$r$ af $f$ efter division med $F$ vha. \nameref{Prop. 5.3.1}.

\subsection{Gröbnerbaser (5.4)}
\label{Gröbnerbaser (5.4)}

\subsubsection{Gröbnerbasis}
\label{Gröbnerbasis}
Lad $R$ være et \nameref{Legeme}.

En mængde af ikke-nul polynomier
\begin{equation*}
  F = (f_1,\ldots,f_m) \subseteq R[X_1,\ldots,X_n]
\end{equation*} 
kaldes en Gröbnerbasis for et \nameref{Ideal} i $R[X_1,\ldots,X_n]$ mht. en
\nameref{Termordning} $\leq$ hvis 
\begin{enumerate}[(i)]
  \item $F \subseteq I$
  \item For alle $f \in I \setminus{0}$,
  \begin{equation*}
    in{\leq}(f_i) \mid in_{\leq}(f)
  \end{equation*}
  for et $i = 1,\ldots,m$.
\end{enumerate}
Mængden $F$ kaldes en Gröbnerbasis mht en termordning $\leq$ hvis den er en
Gröbnerbasis for idealet \nameref{f_1..f_m} mht. $\leq$.

\subsubsection{Prop. 5.4.2}
\label{Prop. 5.4.2}
Lad $R$ være et \nameref{Legeme}. Lad $G = (f_1, \ldots, f_m)$ være en
\nameref{Gröbnerbasis} mht en \nameref{Termordning} $\leq$. For et polynomium
$f \in R[X_1,\ldots,X_n]$ har vi
\begin{equation*}
  f \in I \iff f^G = 0
\end{equation*}
hvor $I =$ \nameref{f_1..f_m}. Se \nameref{Def. 5.3.2} for $f^G$.

\subsubsection{Korollar 5.4.5}
\label{Korollar 5.4.5}
Lad $R$ være et \nameref{Legeme}. Lad $G = (f_1, \ldots, f_m) \subseteq
R[X_1,\ldots,X_n]$ være en \nameref{Gröbnerbasis} for \nameref{Ideal}et $I
\subseteq R$ mht en \nameref{Termordning}. Så er
\begin{equation*}
  I = \text{\nameref{f_1..f_m}}
\end{equation*}

\subsubsection{Prop. 5.4.6}
\label{Prop. 5.4.6}
Lad $R$ være et \nameref{Legeme}. Lad $G = (f_1, \ldots, f_m)$ være en
\nameref{Gröbnerbasis} i $R[X_1,\ldots,X_n]$ mht en \nameref{Termordning}
$\leq$. Så er den entydige rest $r$ i
\begin{equation*}
  f = a_1 f_1 + \cdots + a_m f_m + r
\end{equation*}
ligesom i \nameref{Prop. 5.3.1}, \textit{entydig} for alle $f \in R$.

Resten fra divisionsalgoritmen er uafhængig af rækkefølgen af elementerne $f_1,
\ldots, f_m \in G$.

\subsubsection{Sætning 5.4.7}
\label{Saetning 5.4.7}
Lad $k$ være et \nameref{Legeme}, $\leq$ en \nameref{Termordning} og $I
\subseteq k[X_1,\ldots,X_n]$ et \nameref{Ideal}. Så har $I$ en
\nameref{Gröbnerbasis} mht $\leq$.

\subsubsection{Korollar 5.4.8}
\label{Korollar 5.4.8}
Lad $R$ være et \nameref{Legeme}. Lad $I$ være et arbitrært \nameref{Ideal} i
$R[X_1,\ldots,X_n]$. Så er der endeligt mange polynomier $f_1,\ldots,f_m \in I$
sådan at ethvert polynomium $f \in I$ kan skrives
\begin{equation*}
  f = a_1 f_1 + \cdots + a_m f_m
\end{equation*}
for passende $a_1,\ldots,a_m \in R[X_1,\ldots,X_n]$, hvilket vil sige $I =$
\nameref{f_1..f_m}.

\subsubsection{S-polynomiet}
\label{S-polynomiet}
Def. 5.6.5: S-polynomiet af to ikke-nul polynomier $f$ og $g$ mht en
\nameref{Termordning} $\leq$ er defineret som
\begin{equation*}
  S(f,g) = \frac{X^w}{in_{\leq}(f)}f - \frac{X^w}{in_{\leq}(g)}g
\end{equation*}
hvor $X^w$ er $lcm(in_{\leq}(f), in_{\leq}(g))$.

\subsection{Newton Genvisit (5.5)}
\label{Newton Genvisit (5.5)}
\subsubsection{Sætning 5.5.1}
\label{Saetning 5.5.1}
Lad $f,f_1,\ldots,f_r \in k[X_1,\ldots,X_n]$. Lad $I$ være idealet
\begin{equation*}
  I = \langle T_1 - f_1,\ldots, T_r - f_r \rangle
\end{equation*}
i \nameref{Polynomiumsring}en $A = k[X_1,\ldots,X_n,T_1,\ldots,T_r]$ og $\leq$
være den \nameref{Leksikografisk ordning} givet ved
\begin{equation*}
  T_r \leq \cdots \leq X_n \leq \cdots \leq X_1
\end{equation*}
Lad $G$ være en \nameref{Gröbnerbasis} for $I$ mht $\leq$.

Så kan $f$ skrives som et polynomium i $f_1,\ldots,f_r \iff$
\begin{equation*}
  f^G \in k[T_1,\ldots,T_r]
\end{equation*}
I dette tilfælde er $f = f^G(f_1,\ldots,f_r)$.

\subsection{Buchbergers S-kriterie (5.6)}
\label{Buchbergers S-kriterie (5.6)}
\subsubsection{Korollar 5.6.9}
\label{Korollar 5.6.9}
En sekvens $F = (f_1, \ldots, f_m)$ af polynomier er en \nameref{Gröbnerbasis}
$\iff S(f_i, f_j)^F = 0$ for $1 \leq i \leq j \leq m$. Se \nameref{Def. 5.3.2}
for $S(f_i, f_j)^F$.

\subsection{Buchbergers algoritme (5.7)}
\label{Buchbergers algoritme (5.7)}
\subsubsection{Sætning 5.7.2}
\label{Saetning 5.7.2}
Buchberger's algoritme terminerer og outputtet er en \nameref{Gröbnerbasis}.

\subsection{Den reducerede Gröbnerbasis (5.8)}
\label{Den reducerede Gröbnerbasis (5.8)}
\subsubsection{Minimal Gröbnerbasis}
\label{Minimal Gröbnerbasis}
Def 5.8.1: En minimal gröbnerbasis $(f_1,\ldots,f_m)$ er en
\nameref{Gröbnerbasis} sådan at \begin{enumerate}[(i)]
  \item $in_{\leq}(f_j) \nmid in_{\leq}(f_i)$ for $i \neq j$.
  \item Koefficienten af $in_{\leq}(f_i)$ er 1.  
\end{enumerate}

\subsubsection{Reduceret Gröbnerbasis}
\label{Reduceret Gröbnerbasis}
Def 5.8.2: En reduceret gröbnerbasis $(f_1,\ldots,f_m)$ er en \nameref{Minimal
Gröbnerbasis} hvis $in_{\leq}(f_j) \nmid$ noget led i $f_i$ for $i \neq j$.

\subsubsection{Sætning 5.8.3}
\label{Saetning 5.8.3}
Lad $R$ være et \nameref{Legeme}. Ethvert ideal $I \subseteq R[X_1,\ldots,X_n]$
har en entydig \nameref{Reduceret Gröbnerbasis}.

\subsection{Løsning af ligningssystemer vha Gröbnerbaser (5.9)}
\label{Loesning af ligningssystemer vha Grobnerbaser (5.9)}
\subsubsection{Sætning 5.9.1}
\label{Saetning 5.9.1}
Lad $R$ være et \nameref{Legeme} og $G = (f_1,\ldots,f_m)$ være en
\nameref{Gröbnerbasis} for et \nameref{Ideal} $I \subseteq R[X_1,\ldots,X_n]$
mht den \nameref{Leksikografisk ordning} $\geq$ givet ved $X_n \geq X_{n-1}
\geq \cdots \geq X_1$.

Så er $G \cap R[X_1,\ldots,X_i]$ en gröbnerbasis for idealet $I \cap
R[X_1,\ldots,X_i]$ i $R[X_1,\ldots,X_i]$.

