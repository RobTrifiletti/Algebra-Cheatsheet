\section{Grupper (2)}
\subsection{Indledende gruppeteori (2.1)}
\subsubsection{Komposition}
\label{Komposition}
En komposition på en mængde G er en afbildning $\circ: G \times G \rightarrow G$.
Kompositionen $\circ(g, h)$ skrives ofte $g \circ h$ eller blot $gh$.

\subsubsection{Gruppe}
\label{Gruppe}
Def. 2.1.1: Et par, $(G, \circ)$, bestående af en mængde $G$ og en
\nameref{Komposition} $\circ: G \times G \rightarrow G$ kaldes en
\textit{gruppe} hvis den tilfredsstiller følgende tre egenskaber:
\begin{enumerate}[(i)]
  \item Kompositionen skal være \textbf{associativ}:
  \begin{equation*}
  g_1 \circ (g_2 \circ g_3) = (g_1 \circ g_2) \circ g_3
  \end{equation*}
  for alle $g_1, g_2, g_3 \in G$
  
  \item Der skal eksistere et \textbf{neutralt} element $e \in G$, sådan at:
  \begin{equation*}
  e \circ g = g \quad \text{og}\quad g \circ e = g
  \end{equation*}
  for alle $g \in G$
  
  \item For alle $s \in G$ eksisterer der et \textbf{inverst} element $t \in
  G$, sådan at: 
  \begin{equation*}
  g \circ t = e \quad \text{og} \quad t \circ g = e
  \end{equation*}
\end{enumerate}
En gruppe $G$ kaldes abelsk hvis der for alle $x, y \in G$ gælder at $x \circ y
= y \circ x$. Antallet af elementer $\mid G\mid$ i $G$ kaldes ordnen af $G$.

\subsubsection{Def. 2.1.7}
Lad $g \in G$ være et element i en \nameref{Gruppe}. Så er $g^{-1} \in G$ det
entydige inverse element til $g$.

\subsubsection{Finde inversen til et produkt}
Inversen til et produkt $ab \in G$ er $b^{-1}a^{-1}$ da:
\begin{equation*}
  (ab)(b^{-1}a^{-1}) = a(b(b^{-1}a^{-1})) = a(ea^{-1}) = aa^{-1} = e
\end{equation*}

\subsubsection{Kompositionstabel (2.1.2)}
Se. side 53.

\subsubsection{Gruppen $S_3$}
\label{Gruppen S_3}
Afs. 2.1.4: $S_3$ er \nameref{Gruppe}n bestående af de 6 \nameref{Bijektiv}e
afbildninger der findes for elementerne $X = \{1,2,3\}$. Gruppens komposition
er den normale komposition for afbildninger, dvs. operationen $f(g(x))$, for
bijektionerne $f, g \in G$. \begin{enumerate}[(i)]
  \item Det neutrale element $e$ er identitetsafbildningen $X \rightarrow X$.
  \item Den inverse til en bijektion $f$ er den inverse afbildning $f^{-1}: X
  \rightarrow X$.
  \item Komposition af afbildninger er desuden assosiativ.
\end{enumerate}
Se s. 55 for eksempel og kompositionstabel.

\subsubsection{Multiplikation med $g \in G$ er bijektiv}
Antag $G$ er en \nameref{Gruppe} og $g \in G$. Der eksisterer en afbildning
$\phi:G\rightarrow G$ givet ved $\phi(x) = gx$. Denne afbildning er
\nameref{Bijektiv}.

Vi kan vise $\phi$ er bijektiv, givet dens inverse, nemlig $\psi: G
\rightarrow G$. Denne afbildning er givet ved $\psi(x) = g^{-1}x$. Ved
komposition gælder nu:
\begin{enumerate}[(i)]
  \item $\psi(\phi(x)) = x$
  \item $\phi(\psi(x)) = x$
\end{enumerate}
Udregningerne er sparet væk (se s. 56). Afbildningen $\varepsilon : G
\rightarrow G$, givet ved $\varepsilon(x) = xg$ er også bijektiv og dette vises
på samme måde.

\subsubsection{Prop. 2.1.2}
\label{2.1.2}
Lad $a, b, n \in \Z$. Så gælder:
\begin{enumerate}[(i)]
  \item $a + n\Z = b + n\Z
  \iff a \mycong{b}{n}$

  \item $(a + n\Z) \cap (b + n\Z) = \emptyset
  \iff a \not \mycong{b}{n}$ 
\end{enumerate}
\textit{(i): Hvis $[a]_n = [b]_n$ så er $a$ og $b$ kongruente modulo $n$ og vice
versa. (ii): Hvis $[a]_n$ og $[b]_n$ ingen elementer har tilfælles er de ikke
kongruente modulo $n$ og vice versa.}

\subsection{Undergrupper (2.2)}
\subsubsection{Undergruppe}
\label{Undergruppe}
Def. 2.2.1: En undergruppe af en \nameref{Gruppe} $G$ er en delmængde $H
\subseteq G \neq \emptyset$ sådan at $G$'s \nameref{Komposition} gør $H$ til en
gruppe. 

$H$ er altså en undergruppe af $G$ $\iff$
\begin{enumerate}[(i)]
  \item $e \in H$
  \item $x^{-1} \in H$ for alle $x \in H$
  \item $xy \in H$ for alle $x, y \in H$
\end{enumerate}
\textit{Bemærk at en undergruppe skal være lukket under kompositionen. Per
definition er kompositionen lukket indenfor $G$, den skal også være lukket
indenfor $H$ for at $H$ er en undergruppe.}

\subsubsection{Prop. 2.2.3}
\label{2.2.3}
Lad $H$ være en \nameref{Undergruppe} af $(\Z, +)$. Så gælder:
\begin{equation*}
  H = d\Z = \{ dn \mid n \in \Z \} = \{ \ldots, -2d, -d, 0, d, 2d, \ldots \}
\end{equation*}
for et entydigt naturligt tal $d\in \N$.

\textit{En undergruppe af $\Z$ vil være en gruppe bestående af alle multiplums
af et tal $d \in \N$, ellers kan den f.eks. ikke opfylde lukkethedsegenskaben}.

\subsubsection{Sideklasse}
\label{Sideklasse}
Lad $H$ være en \nameref{Undergruppe} af en \nameref{Gruppe} $G$ og $g \in G$.
Så siger vi: \begin{enumerate}[(i)]
  \item $gH = \{gh \mid h \in H\} \subseteq G$ kaldes en venstre sideklasse.
  Mængden af Hs venstre sideklasse skrives $G\slash H$.
  \item $Hg = \{hg \mid h \in H\} \subseteq G$ kaldes en højre sideklasse.
  Mængden af Hs højre sideklasser skrives $G\backslash H$.
\end{enumerate}
Bemærk samspillet med \nameref{Kvotientgruppe_Z}.

Desuden er $\Z = \Z/3\Z = \{3\Z, 1 + 3\Z, 2 + 3\Z \}$. Se på \nameref{2.2.3} og
lad $G = \Z$ og lad $H = 3\Z$. Dvs. $3\Z$ er en undergruppe til $\Z$. Vi ser nu
at $\Z/3\Z$ er mængden af $3\Z$s venstre sideklasser. Fra \nameref{2.2.7} kan
vi konkludere at dette er lig $\Z$. En sideklasse er en generalisering af en
\nameref{Restklasse}. Mængden af alle restklasser er også lig gruppen den er en
restklasse for. F.eks. er mængden $\{[0]_2,[1]_2 \} = \Z$. Den består af alle
tal der enten har rest 1 eller 0 ved division med 2, hvilket er alle tal i $\Z$.

\subsubsection{Lemma 2.2.6}
Lad $H$ være en \nameref{Undergruppe} af en \nameref{Gruppe} $G$ og lad $x,y \in
G$. Så gælder: 
\begin{enumerate}[(i)]
  \item $x \in xH$
  \item $xH = yH \iff x^{-1}y \in H$ (se \nameref{2.1.2} og tænk på $x^{-1}$ som
  -$x$), dvs. $xH = yH \iff$ $x \mycong{y}{H = d\Z}$.
  \item Hvis $xH \neq yH \Rightarrow xH \cap yH = \emptyset$
  \item Afbildningen $\phi: H \rightarrow H$ givet ved $\phi(h) = xh$ er
  \nameref{Bijektiv}.
\end{enumerate}

\subsubsection{Kor. 2.2.7}
\label{2.2.7}
Lad $H$ være en \nameref{Undergruppe} af $G$. Så gælder:
\begin{equation*}
  G = \bigcup_{g \in G} gH
\end{equation*}
og hvis $g_1 H \neq g_2 H \Rightarrow g_1 H \cap g_2 H = \emptyset$.

\textit{Mængden af Hs venstre \nameref{Sideklasse}r er lig den oprindelige
gruppe. Sideklassenerne $g_1$ og $g_2$ enten er lig hinanden eller disjunkte.}

\subsubsection{Lagranges sætning}
\label{Lagranges saetning}
Sætning 2.2.8: Hvis $H \subseteq G$ er en \nameref{Undergruppe} til en endelig
\nameref{Gruppe} $G$ så:
\begin{equation*}
  \mid G\mid = \mid G/H \mid \mid H\mid
\end{equation*}
\textit{Ordnen af en undergruppe dividerer ordenen af gruppen}.

\subsubsection{Index}
\label{Index}
Def. 2.2.9: Antallet af \nameref{Sideklasse}r $\mid G/H \mid$ kaldes
\textbf{indexet} af $H$ i $G$. Det skrives $[G : H]$.

\subsection{Normale undergrupper (2.3)}
\subsubsection{Komposition for delmængder af en gruppe}
\label{Komp_for_subsets}
Lad $X, Y$ være delmængder af en \nameref{Gruppe} $G$. Så definerer vi
komposition for disse:
\begin{equation*}
  XY = \myset{xy \mid x \in X, y \in Y}
\end{equation*}

\textit{Dette kan vi imidlertid ikke direkte overføre til komposition for
\nameref{Sideklasse}r. Se s. 64 for forklaring. Der skal gælde nedenstående før
vi kan snakke om sideklasser som \nameref{Undergruppe}r}.

\subsubsection{Prop. 2.3.1}
\label{2.3.1}
Lad $H$ være en \nameref{Undergruppe} af $G$. Hvis $gH = Hg$ for alle $g \in G$
så gælder:
\begin{equation*}
  (xH)(yH) = (xy)H
\end{equation*}
for alle $x, y \in G$.

\subsubsection{Normal undergruppe}
\label{Normal undergruppe}
Def. 2.3.2: En \nameref{Undergruppe} $N$ af en \nameref{Gruppe} $G$ kaldes
\textit{normal} hvis
\begin{equation*}
  gNg^{-1} = \{gng^{-1} \mid n \in N\} = N
\end{equation*}
for alle $g \in G$.

\textit{En normal undergruppe $N$ af $G$ tilfredsstiller $gN = Ng$ for
alle $g \in G$. (\nameref{2.3.1}). En \nameref{Undergruppe} af en abelsk gruppe
er altid normal. (gælder ikke begge veje)}.

\textit{Bemærk, hvis $H$ har \nameref{Index} 2 i $G$ er $H$ normal, jf.
opg. 2.15.}

\subsubsection{Kor. 2.3.3}
Lad $N$ være en \nameref{Normal undergruppe} af \nameref{Gruppe} $G$. Så gør
kompositionen \nameref{Komp_for_subsets} $G/N$ til en gruppe og pga.
\nameref{2.3.1} gælder der:
\begin{equation*}
  (g_1 g_2) N = (g_1 N)(g_2 N) 
\end{equation*}
for $g_1 N, g_2 N \in G/N$

\subsubsection{Kvotientgruppe}
\label{Kvotientgruppe}
Def. 2.3.4: Lad $N$ være en \nameref{Normal undergruppe} af en \nameref{Gruppe}
$G$. Så kaldes gruppen bestående af $N$s \nameref{Sideklasse}r, $G/N$, en
kvotientgruppe.

\textit{Husk $G/N$ er mængden af $N$s \nameref{Sideklasse}r og at sideklasser er
en generalisering af restklasser, så passer pengene i forhold til
\nameref{Kvotientgruppe_Z}}.

\subsubsection{Lemma 2.3.6}
Lad $H$ og $K \subseteq G$, hvor $G$ er en \nameref{Gruppe} og $H$ er en
\nameref{Normal undergruppe}. Så er $HK$ en \nameref{Undergruppe} af $G$.

\subsubsection{Kvotientgruppen $\Z/n\Z$}
\label{Kvotientgruppe_Z}
I forlængelse af \nameref{Kvotientgruppe} kigger vi på gruppen $(\Z, +)$. Dette
er en kvotientgruppe hvis elementer er \nameref{Sideklasse}r
(\nameref{Restklasse}r) for et givet $n \in \Z$:
\begin{equation*}
  \Z/n\Z = \myset{[0]_n,[1]_n,\ldots,[n-1]_n}
\end{equation*}
En restklasse er selv en mængde, derfor er en \textbf{kvotientgruppe en mængde
af mængder}.

En restklasse, $[a]_n$ kan udtrykkes på formen:
\begin{equation*}
  a + n\Z = \{a + nx \mid x \in \Z\}
\end{equation*}
hvor $a \in \Z$ er repræsentanten for klassen og $n \in Z$ er det tal vi regner
modulo.

\textit{Et eksempel: $[2]_5 = 2 + 5\Z = \{\ldots, -8, -3, 2, 7, \ldots\}$. Dvs.
mængden af alle tal der er kongruent med 2 modulo 5.}.

Fra \nameref{1.3.2} gælder der nu at hvis $n > 0$:
\begin{equation*}
  a + n\Z = b + n\Z \iff [a]_n = [b]_n
\end{equation*}
\textit{Der eksisterer kun $n-1$ \nameref{Entydig rest}er mht. $n$. Derfor er
$[12]_5 = [2]_5$, da disse har samme rest mht. $n$. Vi bruger derfor som regel
kun repræsentanter fra $0,\ldots,n-1$ da de andre er inkluderede heri.}

Nogle eksempler på specielle kvotientgrupper:
\begin{align*}
  &\Z/0\Z \Rightarrow \myset{x} = x + 0\Z = x  \Rightarrow (\Z, +)\\
  &\Z/1\Z \Rightarrow \myset{[0]_1} = 0 + 1\Z = (\Z, +)
\end{align*}

\subsubsection{Primiske restklasser $(\Z/n\Z)^*$}
\label{Primiske restklasser}
Vi lader mængden:
\begin{equation*}
  (\Z/n\Z)^* = \myset{[a]_n \in \Z/n\Z \mid gcd(a,n) = 1}
\end{equation*}
udgøre alle primiske \nameref{Restklasse}r, hvor $n \in \N$.

\textit{Elementer i $(\Z/n\Z)^*$ er på formen $a + n\Z$, altså det er en
\nameref{Kvotientgruppe}. For at en kvotientgruppe, med \textbf{multiplikation}
med restklasser som komposition, skal være en \nameref{Gruppe} skal alle
elementer $[a]_n \in (\Z/n\Z)^*$ være indbyrdes primiske med $n$. Ordnen af
sådan en gruppe er $\phi(n)$.}

\subsubsection{Supplement til 2.3. Lemma 1}
Lad $G$ være en \nameref{Gruppe}, og lad $H \subseteq G$ være en
\nameref{Undergruppe}. Lad $g \in G$ være et vilkårligt element. Så er mængden
\begin{equation*}
  gHg^{-1} = \myset{ghg^{-1} \mid h \in H}
\end{equation*} 
en undergruppe af G.

\subsubsection{Supplement til 2.3. Lemma 2}
Lad $G$ være en endelig \nameref{Gruppe} af orden $N$, og lad $d$ være en
divisor i $N$. Hvis der findes netop en \nameref{Undergruppe} $H \subseteq G$ af
orden $d$, så er $H$ en \nameref{Normal undergruppe} i $G$.

\subsection{Gruppehomomorfier (2.4)}
\subsubsection{Gruppehomomorfi}
\label{Gruppehomomorfi}
Def. 2.4.1: Lad $G$ og $K$ være \nameref{Gruppe}r. En afbildning $f: G
\rightarrow K$ kaldes en gruppehomomorfi hvis
\begin{equation*}
  f(xy) = f(x)f(y)
\end{equation*}
for alle $x,y \in G$. \nameref{Komposition}en på venstre side af lighedstegnet
er $G$s komposition, mens kompositionen på højre side er $K$s.

\textit{Et eksempel på en gruppehomomorfi er eksponentialfunktionen $e^x :
(\R,+) \rightarrow (\R_{>0}, \cdot)$, hvor $(\R_{>0}, \cdot) = \myset{x \in \R
\mid x > 0}$. Dette er den kendte regel $e^{x + y} = e^x \cdot e^y$ for alle
$x,y \in \R$.}

Der gælder desuden:
\begin{enumerate}[(i)]
  \item $f(e_{G}) = e_H$
  \item $f(g^{-1}) = f(g)^{-1}$
\end{enumerate}

\textit{Altså at det neutrale element i $G$ afbildes til det neutrale element i
$H$, samt inverserne bliver bevaret af afbildningen. Vi siger at en
gruppehomomorfi er kompatibel med gruppestrukturen.}

\subsubsection{Prop. 2.4.9}
\label{Prop. 2.4.9}
Lad $f: G \rightarrow K$ være en \nameref{Gruppehomomorfi}. Så gælder:
\begin{enumerate}[(i)]
  \item Billedet $\text{\nameref{f(G)}} \subseteq K$ er en \nameref{Undergruppe}
  til $K$.
  \item Kernen $\text{\nameref{Ker(f)}} \subseteq G$ er en \nameref{Normal
  undergruppe} til $G$.
  \item $f$ er \nameref{Injektiv} $\iff$ $Ker(f) = \{e\}$.
\end{enumerate}

\subsubsection{Gruppeisomorfi}
\label{Gruppeisomorfi}
\begin{enumerate}[(i)]
    \item En \nameref{Bijektiv} \nameref{Gruppehomomorfi} kaldes en
    gruppeisomorfi.
    \item Vi bruger notationen $f: G \myisomto K$.
    \item Vi skriver $G \myisomorfto K$ og siger $G$ og $K$ er isomorfe.
\end{enumerate}

\subsection{Isomorfisætningen}
\label{isomorfitheorem}
2.5.1: Lad $G$ og $K$ være \nameref{Gruppe}r og $f: G \rightarrow K$ en
\nameref{Gruppehomomorfi} med kernen $N = \text{\nameref{Ker(f)}}$. Så er:
\begin{equation*}
  \myisom{f}: G/N \rightarrow \text{\nameref{f(G)}}
\end{equation*}
givet ved $\myisom{f}(gN) = f(g)$ en \nameref{Veldefineret} afbildning og en
\nameref{Gruppeisomorfi}.

\textit{Da N er en \nameref{Normal undergruppe} til $G$ (\nameref{Prop. 2.4.9})
er $G/N$ en \nameref{Kvotientgruppe}. Denne indeholder alle
\nameref{Sideklasse}r for N. $\myisom{f}$ afbilder fra disse sideklasser til
billedet af $f$ (dvs. alle elementer i $K$ der bliver afbildet af $f$). Man
finder oftest først en \nameref{Surjektiv} \nameref{Gruppehomomorfi} $f: G
\rightarrow K$ for en passende gruppe $K$, sådan at $N = ker(f)$, så giver
sætningen gruppeisomorfien $\myisom{f}$.}

\subsection{Orden af et element i en gruppe (2.6)}
\subsubsection{Prop. 2.6.1}
\label{2.6.1}
Lad $G$ være en \nameref{Gruppe} og $g \in G$. Afbildningen
\begin{equation*}
  f_g: \Z \rightarrow G
\end{equation*}
givet ved $f_g(n) = g^n$ er en \nameref{Gruppehomomorfi} fra $(\Z, +)$ til $G$.

\textit{Denne afbildning er den kendte potensopløftning (med + som komposition),
og det er en \nameref{Gruppehomomorfi}.}

\subsubsection{Ordnen af et element}
\label{ord(g)}
\begin{enumerate}[(i)]
  \item Billedet $f_g(\Z) = \{g^n \mid n \in \Z\}$ skrives $\langle g \rangle$.
  \item $\langle g \rangle$ er en abelsk \nameref{Undergruppe} af $G$.
  \item Antallet af elementer i $\langle g \rangle$ kaldes ordnen af $g$ og
  skrives $ord(g)$.
  \item $ord(g)$ kan man tænke på som den mindste positive potens af $g$ der
  giver $e$.
  \item Hvis sådan en potens ikke findes siges $g$ at have uendelig orden.
\end{enumerate}

\subsubsection{Prop. 2.6.3}
Lad $G$ være en endelig \nameref{Gruppe} og lad $g \in G$. Så gælder:
\begin{enumerate}[(i)]
  \item $ord(g)$ dividerer $\mid G\mid$
  \item $g^{\mid G\mid} = e$
  \item Hvis $g^n = e$ for et $n > 0$ så $ord(g)\mid n$.
\end{enumerate}

\subsubsection{Supplement til 2.6}
Lad $G$ være en \nameref{Gruppe} og $g \in G$. Betragt afbildningen $f_g: \Z
\rightarrow G$ (\nameref{2.6.1}). Der findes nu et heltal $n_g > 0$ sådan at
$Ker(f_g) = n_g\Z$ (\nameref{2.2.3}). Fra dette kan vi konkludere
\begin{enumerate}[(i)]
  \item Hvis $n_g = 0$ er $f_g$ \nameref{Injektiv}.
  \item Hvis $n_g > 0$, så er $g_n = g_m \iff n \mycong{m}{n_g}$. Dette medfører
  at
  \begin{equation*}
  \langle g \rangle = \myset{g^0 = e, g^1 = g, g^2, \ldots, g^{n_g -1}}
  \end{equation*}
\end{enumerate}

Der gælder desuden at
\begin{equation*}
  f(g_n) = f(g)^n \quad \text{for alle g $\in G$ og alle $n \in \Z$}
\end{equation*}

\textit{Hvis $Ker (f_g) = {0}$ er det kun $0 \in \Z$ der afbilder til $e
\in G$ og derved afbilder de andre elementer i $\Z$ til forskellige elementer i
$G$. Hvis $n_g > 0$ så afbilder kongruente heltal $\in \Z$ til det same element
i $G$. Derved giver det kun mening at potensopløfte elementer i $g$ med tal fra
$[0;n_g[$, da billedet af $f_g$ kun består af multiplum as disse.}

\subsection{Cykliske grupper (2.7)}
\subsubsection{Cyklisk gruppe}
\label{Cyklisk gruppe}
Def. 2.7.1: En \nameref{Gruppe} $G$ siges at være cyklisk hvis den indeholder et
element $g$, sådan at $G = \langle g \rangle$. Elementet $g$ kaldes en
frembringer for $G$ og vi siger $g$ frembringer $G$.

\textit{Hvis alle elementer i $G$ kan skrives som potenser af $g$ er $G$
cyklisk.}

\subsubsection{Prop 2.7.2}
Lad $G$ være en \nameref{Gruppe} med primtalsorden $\mid G\mid = p$. Der gælder
nu at $\Z/p\Z \myisomorfto G$, som er en \nameref{Cyklisk gruppe}.

\textit{Dvs. hvis $G$ har en primtalsordenen $p$, så eksisterer der en
gruppeisomorfi $f: \Z/p\Z \rightarrow G$. Dette kommer af
\nameref{isomorfitheorem}}

\subsubsection{Prop 2.7.4}
Lad $G$ være en \nameref{Cyklisk gruppe}. Så gælder:
\begin{enumerate}[(i)]
  \item Enhver \nameref{Undergruppe} af $G$ er cyklisk.
  \item Antag $G$ er endelig og at $d$ er en divisor i $\mid G\mid$. Så
  indeholder $G$ en entydig undergruppe $H$, hvor $\mid H\mid = d$.
  \item Der er $\phi(d)$ elementer af orden $d$ i $G$. Disse er frembringerne
  for $H$.
\end{enumerate}
\textit{Eksempel: Lad $G = \Z/6\Z =
\myset{[0]_6,[1]_6,[2]_6,[3]_6,[4]_6,[5]_6}$. Det ses at $ord(G) = 6$, vi lader
derfor $d = 3$. En undergruppe skal indeholde det neutrale element $[0]_6$, så
denne skal være indeholdt.}

\textit{Dernæst skal hvert element i en undergruppe have en invers. Vi ved også
at enhver undergruppe af $G$ er cyklisk, de to elementer der tilfredsstiller
disse krav er $[2]_6$ og $[4]_6$. Dvs. $H = \myset{[0]_6,[2]_6,[4]_6}$.}

\textit{(iii) siger nu at der skal være 2 elementer i $G$ af ordnen 3. Det
første element der tilfredsstiller dette er $[2]_6$, da $[2]_{6}^3 = [0]_6$.
Det andet element er $[4]_6$, da $[4]_{6}^3 = [0]_6$. Ingen af de andre
elementer har denne egenskab.
(iii) siger nu at $[2]_6$ og $[4]_6$ genererer netop $H$.}

\subsubsection{Kor. 2.7.6}
Lad $N > 0 \in \Z$. Så gælder:
\begin{equation*}
  \sum_{d\mid N} \phi(d) = N
\end{equation*}
hvor der summeres over $d \in div(N)$.

\textit{Eksempel: Lad $N = 6$. $div(N) = div(6) = \myset{1,2,3,6}$. Nu siger
korollaret at $N = \phi(1) + \phi(2) + \phi(3) + \phi(6) = 1 + 1 + 2 + 2 = 6$.}


\subsection{Grupper og tal (2.8)}
\subsubsection{Produktgrupper}
Hvis $G_1, G_2,\ldots,G_n$ er \nameref{Gruppe}r, så har produktet:
\begin{equation*}
  G = G_1 \times G_2 \times \cdots \times G_n = \{(g_1,g_2,\ldots,g_n) \mid g_1
  \in G_1, g_2 \in G_2,\ldots g_n \in G_n \}
\end{equation*}
den naturlige \nameref{Komposition}

\begin{equation*}
  (g_1, g_2,\ldots,g_n)(h_1,h_2,\ldots,h_n) = (g_1 h_1,g_2 h_2, \ldots, g_n h_n)
\end{equation*}
\textit{En produktgruppe skal ses som en gruppe indeholdende tupler, bestående
af et element fra hver faktorgruppe. To produktgrupper $G$ og $H$ komponeres
som vist ovenover.}

\subsubsection{Lemma 2.8.1}
Lad $M, N$ være \nameref{Normal undergruppe}r af en \nameref{Gruppe} $G$, hvor
$M \cap N = \{e\}$. Så er: \begin{enumerate}[(i)]
  \item $MN$ en \nameref{Undergruppe} til $G$.
  \item $\pi: M \times N \rightarrow MN$, givet ved $\pi(x,y) = xy$, en
  \nameref{Gruppeisomorfi}.
\end{enumerate}


\subsection{Symmetriske gruppe (2.9)}
\textit{Den symmetriske gruppe $S_n$ har som elementer alle permutationer af
\nameref{Bijektiv}e afbildninger der afbilder fra tupler bestående af $n$
symboler til sig selv. Altså en gruppe bestående af alle kombinationer af
bijektioner over de $n$ symboler. Elementerne i $S_n$ kaldes permutationer.
\nameref{Gruppen S_3} er et eksempel på en symmetrisk gruppe.}

Der gælder at $\mid S_n\mid = n!$

\subsubsection{Def. 2.9.1}
Antag $\sigma \in S_n$. Så definerer vi
\begin{equation*}
  M_\sigma = \{x \in M_n \mid \sigma(x) \neq x\}
\end{equation*}
hvor $M_n = \{ 1, 2, \ldots, n \}$. Permutationerne $\sigma, \tau \in S_n$
kaldes disjunkte hvis $M_\sigma \cap M_\tau = \emptyset$.

\textit{$\sigma$ og $\tau$ er elementer i $S_n$, altså \nameref{Bijektiv}e
afbildninger over $M_n$. $M_\sigma$ og $M_\tau$ er mængderne bestående af de
tal som henholdsvist $\sigma$ og $\tau$ permuterer. $M_\sigma \cap M_\tau =
\emptyset$ hvis $\sigma$ og $\tau$ permuterer forskellige tal}.

\subsubsection{Prop 2.9.2}
Lad $\sigma, \tau \in S_n$ være disjunkte permutationer. Så gælder:
\begin{equation*}
  \sigma \tau = \tau \sigma
\end{equation*}
samt at $M_{\sigma\tau} = M_\sigma \cup M_\tau$

\subsubsection{$k$-cykel}
\label{$k$-cykel}
Antag vi er givet $k$ forskellige elementer i $M_n$. En permutation $\sigma \in
S_n$, givet ved:
\begin{equation*}
  \sigma(x_1) = x_2, \quad \sigma(x_2) = x_3, \quad \ldots, \quad \sigma_{k-1} =
  x_k, \quad \sigma(x_k) = x_1
\end{equation*}
og $\sigma(x) = x$ hvis $x \nin \{x_1,\ldots,x_k\}$ kaldes en $k$-cykel. Sådan
en cykel skrives $\sigma(x_1 x_2 \ldots x_k)$.

Bemærk at $M_\sigma = \{x_1, x_2, \ldots, x_k \}$ og at ordnen af en $k$-cykel i
$S_n$ er $k$. Se \nameref{ord(g)}.

\subsubsection{Ordnen af en cykel}
\label{Orden af en cykel}
Ordnen af en \nameref{$k$-cykel} er $k$. Ordnen er altså lig længden af cyklen.
\textit{Fra Wikipedia}.

\textit{Altså er en $k$-cykel en permutation der gør at alle $k$ elementer
bliver flyttet en frem og hvor det sidste element bliver flyttet til starten. De
andre ikke-k elementer i $M_n$ permuterer en $k$-cykel ikke. Tænk linked
list.}
 
\subsubsection{Prop 2.9.5}
Lad $\sigma \in S_n$ være skrevet som et produkt af disjunkte cykler $\sigma_1
\cdots \sigma_r$. Så er $ord(\sigma) = lcm(ord(\sigma_1), \cdots,
ord(\sigma_r))$. Se \nameref{ord(g)}.

\subsubsection{Prop 2.9.6}
Alle permutationer $\sigma \in S_n$ er et produkt af entydige disjunkte cykler.

\subsubsection{Lemma 2.9.8}
Antag $(i_1 i_2 \ldots i_k)$ er en \nameref{$k$-cykel} og at $\sigma \in
S_n$ er en vilkårlig permutation. Så gælder:
\begin{equation*}
  \sigma(i_1 i_2 \ldots i_k)\sigma^{-1} =
  (\sigma(i_1)\sigma(i_2) \ldots \sigma(i_k))
\end{equation*}
\textit{Lemmaet fortæller, hvordan man beregner en \nameref{Konjugering} af en
$k$-cykel. Eksempel: Hvis vi vil konjugere 3-cyklen $(1 2 3)$ med permutationen
$\sigma = (1 2 5)(3 4)$ i $S_5$, dvs hvis vi vil udregne $\sigma$ (1 2 3)
$\sigma^{-1}$, så behøver vi blot at finde $\sigma(1)=2$, $\sigma(2)=5$ og
$\sigma(3)=4$. Lemmaet siger nu at $\sigma (1 2 3) \sigma^{-1} = (\sigma(1)
\sigma(2) \sigma(3)) = (2 5 4)$. Vi slipper altså for at beregne $\sigma^{-1}$.}

\subsubsection{Transposition}
\label{Transposition}
En 2-cykel kaldes en transposition i $S_n$. Pga. det er en 2-cykel er en
transposition sin egen invers.

\subsubsection{Simpel transposition}
\label{Simpel transposition}
En simpel transposition er en \nameref{Transposition} på følgende form:
\begin{equation*}
  s_i = (i \quad i + 1)
\end{equation*}
hvor $i = 1,\ldots, n-1$

\textit{Altså er en simpel transposition en transposition der permuterer et
symbol $i$, en plads frem. $i$ kan ligge i intervallet $1 \leq i < n $.}

\subsubsection{Bubble sort}
Simpel sorteringsalgoritme. Går igennem elementerne en ad gangen og sammenligner
naboparrene, hvis et efterfølgende element er skarpt større end det forrige
swappes disse og algoritmen starter forfra. Når algoritmen når til sidste
element er elementerne sorterede. Disse swaps er eksempler på simple
transpositioner.

\subsubsection{Inversion}
\label{Inversion}
Def. 2.9.10: Lad $\sigma \in S_n$ være en permutation. Et par af indekser $(i,
j)$, hvor $1 \leq i < j \leq n$, kaldes en \textbf{inversion} af $\sigma$, hvis
$\sigma(i) > \sigma(j)$.

\textit{Et par $(i, j)$, hvor $i$ er mindre end $j$, men hvor permutationen
$\sigma$ gør at $\sigma(i) > \sigma(j)$ kaldes en inversion af $\sigma$.}

Lad
\begin{equation*}
  I_\sigma = \myset{(i,j) \mid 1\leq i < j \leq n \myand \sigma(i) > \sigma(j)}
\end{equation*}
være mængden af alle inversioner og lad $n(\sigma) = \mid I_\sigma\mid$ være
antallet af inversioner af $\sigma$.

\subsubsection{Prop 2.9.12}
Permutationen $\sigma \in S_n$ er identitetsafbildningen $\iff n(\sigma) = 0$.
Hvis $\sigma$ ikke er identitetsafbildningen så eksisterer der et $1 \leq i < n$
sådan at $\sigma(i) > \sigma(i + 1)$.

\textit{Identitetsafbildningen er den eneste permutation der ingen
\nameref{Inversion}er har. Hvis $\sigma$ ikke er identitetsafbildningen så
eksisterer der en inversion af $\sigma$ for et nabopar $(i, j)$, altså hvor $j
= i + 1$.}

\subsubsection{Lemma 2.9.13}
Lad $s_i \in S_n$ være en \nameref{Simpel transposition} og $\sigma
\in S_n$. Så gælder:
\begin{equation*}
n(\sigma s_i) = \left\{
\begin{array}{l l}
n(\sigma) + 1 & \quad \text{hvis $\sigma(i) < \sigma(i + 1)$}\\
n(\sigma) - 1 & \quad \text{hvis $\sigma(i) > \sigma(i + 1)$}\\
\end{array} \right.
\end{equation*}

\textit{Lemmaet siger at antallet af \nameref{Inversion}er af $\sigma$
komponeret med den simple transposition $s_i$, er lig $n(\sigma) \pm 1$
afhængigt af om $(i, i + 1)$ er en inversion af $\sigma$ eller ej.}

\subsubsection{Lemma 2.9.14}
Lad $\sigma \in S_n$. Så gælder:
\begin{enumerate}[(i)]
  \item $\sigma$ er et produkt af $n(\sigma)$ \nameref{Simpel transposition}er.
  \item $n(\sigma)$ er det mindste antal simple transpositioner der skal bruges
  for at skrive $\sigma$ som et produkt af simple transpositioner.
\end{enumerate}

\subsubsection{Def. 2.9.15}
Fortegnet af en permutation $\sigma \in S_n$ er
\begin{equation*}
  sgn(\sigma) = (-1)^{n(\sigma)}
\end{equation*}
En permutation med positivt fortegn kaldes lige, en permutation med negativt
fortegn kaldes ulige.

\textit{Antallet af \nameref{Inversion}er af en permutation afgør om den er
lige eller ulige.}

\subsubsection{Prop 2.9.16}
\label{2.9.16}
Afbildningen
\begin{equation*}
  sgn: S_n \rightarrow \myset{\pm 1}
\end{equation*}
er en \nameref{Gruppehomomorfi}, hvor \nameref{Komposition}en i $\myset{\pm 1}$
er multiplikation.

\subsubsection{Den Alternerende gruppe}
\label{Den Alernerende gruppe}
Mængden af lige permutationer i $S_n$ skrives $A_n$ og kaldes den alternerende
\nameref{Gruppe}. Fra \nameref{2.9.16} og \nameref{Prop. 2.4.9} (ii) kan vi
konkludere at $A_n$ er en \nameref{Normal undergruppe} til $S_n$, da $A_n$ netop
er kernen til afbildningen $sgn$. Der gælder at $\mid A_n\mid = n! / 2$.

\subsubsection{Prop 2.9.17}
Lad $n \geq 2$. 
\begin{enumerate}[(i)]
  \item En \nameref{Transposition} $\tau = (i \quad j) \in S_n$ er en ulige
  permutation.
  \item Fortegnet af en r-cykel $\sigma = (x_1 x_2 \ldots x_r)$ er $(-1)^{r-1}$.
\end{enumerate}

\subsubsection{Simpel gruppe}
\label{Simpel gruppe}
En \nameref{Gruppe} $N$ kaldes simpel hvis $\myset{e}$ og $N$ er de eneste
\nameref{Normal undergruppe}r af $N$.

\subsubsection{Lemma 2.9.18}
Enhver permutation i $A_n$ er et produkt af en 3-cykel hvis $n \geq 3$.

\subsubsection{Sætning 2.9.19}
\nameref{Den Alernerende gruppe} $A_n$ er \nameref{Simpel gruppe} for $n \geq
5$.

\subsubsection{Lemma 2.9.20}
Enhver 3-cykel er et produkt af en simpel 3-cykler i $A_n$ hvis $n \geq 3$.

En 3-cykel kaldes simpel hvis den er på formen $(k \quad k + 1 \quad k + 2)$.

\subsection{Gruppevirkninger (2.10)}
\subsubsection{Gruppevirkning}
\label{Gruppevirkning}
Def 2.10.1: Lad $G$ være en \nameref{Gruppe} og $S$ en mængde. Vi siger at $G$
virker på en gruppe hvis der eksisterer en afblidning
\begin{equation*}
  \alpha: G \times S \rightarrow S
\end{equation*}
som skrives $\alpha(g,s) = g \cdot s$ sådan at:
\begin{enumerate}[(i)]
  \item $e \cdot s = s$ for alle $s \in S$
  \item $(g \cdot h) s = g (h \cdot s)$ for alle $g,h \in G$ og for alle $s \in
  S$.
\end{enumerate}

\textit{En virkning er en afbildning der afbilder fra et element $s \in S$ og et
gruppeelement $g \in G$, til $S$. Denne afbildning skal respektere det neutrale
element og være associativ.}

\subsubsection{Def. 2.10.2}
\label{2.10.2}
Lad $\alpha: G \times S \rightarrow S$ være en \nameref{Gruppevirkning} af $G$
på $S$ og $s \in S$ et element i $S$. Så er:
\begin{equation*}
  G \cdot s = Gs = \myset{gs\mid g \in G}
\end{equation*}
\textbf{banen} af $s$ under virkningen af $G$. Mængden af baner
$\myset{Gs\mid s \in S}$ skriver vi $S/G$.

\textit{Banen af $s$ under $G$  $\subseteq S$ er den delmængde af $S$ som
$\alpha$ afbilder dette $s$ og hele $G$ til.}

Lad $X \subseteq S$ være en delmængde af $S$ og $g \cdot X = gX = \myset{gx\mid
x \in X}$, hvor $g \in G$. Så er:
\begin{equation*}
  G_X = \myset{g \in G\mid gX = X}
\end{equation*}
\textbf{stabilisatoren} for $X$. Hvis $X = \myset{x}$ skrives $G_X$ som
$G_x$.

\textit{Stabilisatoren $G_X \subseteq G$ er den delmængde hvis virkning
ikke ændrer elementerne i $X$}.

Et \textbf{fikspunkt} for en virkning er et element $s \in S$ sådan at $gs = s$
for alle $g \in G$. Mængden af fikspunkter skriver vi $S^G$.

\textit{Et fikspunkt er et element $s \in S$ der er neutral mht. $G$s virkning.}

\subsubsection{Prop. 2.10.5}
Lad $\alpha: G \times S \rightarrow S$ være en \nameref{Gruppevirkning}. Så
gælder: \begin{enumerate}[(i)]
  \item Lad $X \subseteq S$ være en delmængde af $S$. Så er $G_X$ en
  \nameref{Undergruppe} af $G$.
  \item Mængden $S$ er foreningen af $G$-baner:
  \begin{equation*}
  S = \bigcup_{s \in S}Gs
  \end{equation*}
  hvor $Gs \neq Gt \Rightarrow Gs \cap Gt = \emptyset$ hvis $s, t \in S$.
  \item Lad $x \in S$. Så er
  \begin{equation*}
  \tilde{f}: G/G_x \rightarrow Gx
  \end{equation*}
  givet ved $\myisom{f}(gG_x) = gx$ en \nameref{Veldefineret} \nameref{Bijektiv}
  afbildning fra $G_x$s venstre \nameref{Sideklasse}r til banen $Gx$.
\end{enumerate}

\subsubsection{Kor. 2.10.7}
\label{2.10.7}
Lad $G \times S \rightarrow S$ være en \nameref{Gruppevirkning} hvor $S$ er
endelig. Så gælder:
\begin{equation*}
  \mid S\mid = \mid S^G\mid + \sum_x \mid G/G_x\mid
\end{equation*}
hvor der summes ved at vælge et element $x$ fra hver bane med mere end et
element.

\textit{Ordnen af $S$ er lig antallet af fikspunkter + summen af antallet af
venstre \nameref{Sideklasse}r for hvert element i $G_x$, altså de elementer der
er stabile under \nameref{Konjugering} med $x$.}

\subsubsection{Lemma 2.10.8 (Burnside)}
Lad $G \times S \rightarrow S$ være en \nameref{Gruppevirkning}, hvor $G$ er en
endelig \nameref{Gruppe} og $S$ en endelig mængde. Så gælder:
\begin{equation*}
  \mid S/G\mid = \frac{\sum_{g \in G}\mid S^g\mid}{\mid G\mid}
\end{equation*}
hvor $S^g = \myset{x \in S\mid gx = x}$.

\textit{$S^g$ er de elementer i $S$ der er fikspunkter under $g$. Antallet af
baner for en virkning er altså lig summen af antallet af fikspunker under $g$
divideret med ordnen af $G$. Sagt med andre ord, antallet af baner (et
naturligt tal eller $+\infty$) er lig det gennemsnitlige antal fikspunkt under
et element af $g$.}

\subsubsection{Konjugering}
\label{Konjugering}
Afbildningen $\alpha: G \times G \rightarrow G$ givet ved $\alpha(g,h) =
ghg^{-1}$ er en \nameref{Gruppevirkning} af G på G. Den kaldes konjugering.

$C(h)$ kaldes konjugeringsklassen indeholdende $h$ og udgør banen
(\nameref{2.10.2})
\begin{equation*}
  C(h) = G \cdot h = \myset{ghg^{-1}\mid g\in G}
\end{equation*}

\textit{Altså er konjugeringsklassen indeholdende $h$ en delmængde af $G$
indeholdende de elementer som $ghg^{-1}$ afbilder til. I en abelsk gruppe er
enhver konjugeringsklasse en singleton}.

$Z(h)$ kaldes centralisatoren af $h$ og udgør stabilisatoren (\nameref{2.10.2})
$G_h$:
\begin{equation*}
  Z(h) = \myset{g \in G \mid gh = hg}
\end{equation*}

\textit{Altså er centralisatoren af $h$ mængden af elementer i $G$ hvor
$ghg^{-1} = hgh^{-1}$.}

$Z(G)$ kaldes centeret af $G$ og udgør mængden af fikspunkter for $G$:
\begin{equation*}
  Z(G) = G^G = \myset{g \in G\mid gx = xg \text{ for alle }x \in G}
\end{equation*}

\textit{$Z(G)$ er altså mængden af elementer der er stabile med alle elementer i
$G$ under konjugering. $Z(G)$ er en normal abelsk undergruppe af $G$ og
indeholder mindst elementet $e$ for konjugeringsvirkningen.}

Hvis $G$ er en endelig \nameref{Gruppe} kan vi skrive \nameref{2.10.7} som:
\begin{equation*}
  \mid G\mid = \mid Z(G)\mid + \sum_{h \in G}\mid G/Z(h)\mid
\end{equation*}

\textit{Ordnen af $G$ er lig antallet af fikspunkter for $G$ + summen af
antallet af venstre \nameref{Sideklasse}r for elementerne i $G_h$,
altså de elementer der er stabile under konjugering med $h$.}

$N_G(H)$ kaldes normalisatoren af $H$ i $G$ og udgør stabilisatoren af
\nameref{Undergruppe}n $H \subseteq G$:
\begin{equation*}
  N_G(H) = G_H = \myset{g \in G \mid gHg^{-1} = H}
\end{equation*}

\textit{$H$ er en \nameref{Normal undergruppe} $\iff N_G(H) = G$}.

\subsubsection{$p$-gruppe}
\label{p-gruppe}
En endelig \nameref{Gruppe} med orden $p^r$, hvor $p$ er et \nameref{Primtal} og
$r \in \N$ kaldes en $p$-gruppe.

\subsubsection{Prop. 2.10.13}
Lad $G$ være en ikke-triviel \nameref{p-gruppe} med en \nameref{Gruppevirkning}
på en mængde $S$. Så er
\begin{equation*}
  \mid S\mid \mycong{\mid S^G\mid}{p}
\end{equation*}

\textit{Ordnen af $S$ er kongruent $(mod$  $p)$ til antallet af fikspunkter i
$S$ under $G$, hvis $G$ er en ikke-triviel $p$-gruppe.}

\subsubsection{Kor. 2.10.14}
Lad $G$ være en ikke-triviel \nameref{p-gruppe} med orden $p^r$, så er
\begin{equation*}
  \mid G\mid \mycong{\mid Z(G)\mid}{p}
\end{equation*}
og $\mid Z(G)\mid > 1$.

\textit{Ordnen af $G$ er kongruent $(mod$  $p)$ til størrelsen af centeret af
$G$ (\nameref{Konjugering}), altså antallet af fikspunkter for $G$, samt
centeret af $G$ består af mere end 1 element.}

\subsubsection{Kor. 2.10.15}
Lad $p$ være et \nameref{Primtal}. En gruppe $G$ med orden $\mid G\mid = p^2$ er
abelsk.

\subsubsection{Sylow $p$-undergruppe}
\label{Sylow p-undergruppe}
Def. 2.10.16: Lad $G$ være en endelig \nameref{Gruppe} og $p$ et
\nameref{Primtal}. Antag nu at $\mid G\mid = p^r m$, hvor $p \nmid m$. En
\nameref{Undergruppe} $H \subseteq G$ med orden $\mid H\mid = p^r$ kaldes en
Sylow $p$-undergruppe.

\textit{Hvis en \nameref{p-gruppe} $H \subseteq G$ og $\mid G\mid = p^r m$,
hvor $p \nmid m$, kaldes $H$ en Sylow $p$-undergruppe.}

\subsubsection{Sætning 2.10.17 (Første Sylow Sætning)}
Lad $G$ være en endelig \nameref{Gruppe} og $p$ et \nameref{Primtal}. Antag at
$\mid G\mid = p^r m$, hvor $p \nmid m$. Så indeholder $G$ en \nameref{Sylow
p-undergruppe}.

\subsubsection{Sætning 2.10.18 (Anden Sylow Sætning)}
Lad $G$ være en endelig \nameref{Gruppe} og $P, Q$ to \nameref{Sylow
p-undergruppe}r. Så eksisterer der et $g \in G$ sådan at:
\begin{equation*}
  gPg^{-1} = Q
\end{equation*}
Ydermere, enhver $p$-undergruppe $H$ (\nameref{p-gruppe}) er indeholdt i en
Sylow $p$-undergruppe.

\textit{Hvis $P, Q$ er to \nameref{Sylow p-undergruppe}r, har de samme orden. Da
der eksisterer et $g \in G$ sådan at $gPg^{-1} = Q$ siger vi at $P$ og $Q$
tilhører samme konjugeringsklasse (\nameref{Konjugering}).}

\subsubsection{Sætning 2.10.19 (Tredje Sylow Sætning)}
Lad $G$ være en endelig \nameref{Gruppe} med orden $p^r m$, hvor $p \nmid m$.
Lad $Syl_p (G)$ være mængden af \nameref{Sylow p-undergruppe}r. Så gælder:
\begin{enumerate}[(i)]
  \item $\mid Syl_p (G)\mid$ dividerer m.
  \item $\mid Syl_p (G)\mid \mycong{1}{p}$
\end{enumerate}
Eksempler på brugen af Sylow sætningerne kan ses på side 103.

\subsubsection{Supplement til 2.10. Lemma 1}
Lad $G$ være en \nameref{Gruppe} og $g \in G$. Afbildningen 
\begin{equation*}
  i_g: G \rightarrow G
\end{equation*}
givet ved $i_g (x) = gxg^{-1}$ kaldes \nameref{Konjugering} med $g$.

Afbildningen $i_g$ er en \nameref{Gruppeisomorfi}. Dette medfører:
\begin{enumerate}[(i)]
  \item Hvis $H$ er en \nameref{Undergruppe} af $G$, så er $i_g (H)$ en
  undergruppe af $G$.
  \item $ord(x) = ord(gxg^{-1})$ for alle $x \in G$.
\end{enumerate}

\subsubsection{Supplement til 2.10. Korrolar}
Lad $p$ være et \nameref{Primtal} og lad $G$ være en endelig \nameref{p-gruppe}
med $\mid G\mid> p$. Så er $G$ ikke en \nameref{Simpel gruppe}.

\subsubsection{Supplement til 2.10. Observation}
Lad $G$ være en endelig \nameref{Gruppe} og $p$ et \nameref{Primtal}. Hvis $G$
kun har een \nameref{Sylow p-undergruppe} $P$, så er $P$ en \nameref{Normal
undergruppe} af $G$.

\subsubsection{Supplement til 2.10. Lemma 2}
Lad $G$ være en endelig \nameref{Gruppe} og lad $p$ og $q$ være to entydige
\nameref{Primtal}. Lad $P$ være en \nameref{Sylow p-undergruppe} af $G$ og $Q$
en Sylow $q$-undergruppe af $G$. Så gælder:
\begin{enumerate}[(i)]
  \item $P \cap Q = \myset{e}$
  \item Hvis $P$ og $Q$ er \nameref{Normal undergruppe}r i $G$, så er $xy = yx$
  for alle $x \in P$ og $y \in Q$
  \item Hvis $p$ og $q$ er de eneste primdivisorer af $\mid G\mid$, og hvis
  enten $P$ eller $Q$ er normale i $G$, så er $G = PQ$.
\end{enumerate}